\section{Beginning}\label{sec:beginning}
The prehistory of the basic concept of working of gyrotron dates to the twentieth century where scientists tried to generate electromagnetic waves by radiation of oscillating electrons .Theoretically, the research on this phenomenon was started by 
The gyrotron was first experimented by Gaponov and Pental in the 1959 based on the research in Electron Cyclotron Resonance by Twiss in Australia, Schneider in the USA and Gaponov in the USSR, whose results were then published by scientists of USA and USSR.\cite{ref:tg}\cite{ref:gt}
 
A gyrotron was constructed at the Tadio-Physical Institute, Gorki, USSR in 1963 consisting of:

\begin{enumerate}
\item Annular Magnetron electron gun
\item Adiabatic Compressor for the rotating stream of electrons
\item Smooth walled cavity
\end{enumerate}

The first working gyrotron was constructed by Hirshfield and Wachtel in 1964. This gyrotron supplied 6W of power at 10GHz frequency in the CW mode. MIG (Magnetron Injection Gun) replaced the annular magnetron gun along with a single cavity which resulted in higher efficiency and increased power output.

\section{Development}
The progress in the development of the gyrotron gained in 1980s, both in the theoretical and experimental aspects including other devices based on the same principle like gyro-TWT, gyro-klystron  etc. This era of development primarily dealt with elimination of of parasitic oscillation, proper profiling of the cavity, increasing the work frequency and the development of tunable gyrotrons. This led to the including the helical output launcher developed by Vlasow in the same period. Many countries like France, Korea, Germany, Australia joined the research building their own gyrotrons later during the 1980s.

\section{Current Stage}
 Recently, China and India started researching and developing their own gyrotrons with a drastic improvement occurred in the development of tetrahertz and co-axial gyrotrons. The most significant contribution to the current state of gyrotrons is by ITER (International Thermonuclear Experimental Reactor), an international nuclear fusion research and engineering project led by sevene entities - European Union, India, Japan, China, Russia, South Korea, and the United States to build the world's largest magnetic plasma confinement experiment.

