\setcounter{page}{1}
\thispagestyle{empty}
Gyrotrons are high-power devices based on vacuum tubes that produce millimeter-wave electromagnetic waves. Their operating principle is based on Electron Cyclotron Resonance, a phenomenon where stimulated cyclotron radiation of electrons oscillating in a strong magnetic field transfer their energy to the field. These devices operate in the frequency range 10 - 95 GHz and researchers are still trying to find ways to expand this range to terahertz(THz). Gyrotrons can provide power from 10kW to 2MW and can be used in both continuous wave and short-pulse modes.\\

Gyrotrons are termed as fast-wave devices because the size of the interaction structure of these devices exceed the wavelength of the radiation generated unlike slow-wave devices, whose interaction structures are of the order of the wavelength of the waves generated, by them.\\

Gyrotrons are capable of generating high power without causing much damage to the structure due to the large amount of heat produced because the interaction cavity inside the gyrotrons are made of copper making them easier to maintain and reduce their temperature.
